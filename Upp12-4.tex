\documentclass[12pt]{article}
\usepackage[swedish]{babel}
\usepackage[utf8]{inputenc}                                  
\usepackage{enumitem, tasks}
\usepackage{amsmath,amssymb,amsthm, amsfonts}

\begin{document}
%<*Question>
Arrangören av en konsert säljer biljetter till det givna priset $500:-$. Antalet köpare beror på hur mycket han lägger ut reklam enligt $Q(R)=1000+4\sqrt{R}$, där $Q$ betecknar antalet sålda biljetter och $R$ utgifterna på reklam. Antag att arrangören maximerar vinsten och bestäm hur mycket pengar som då kommer att satsas på reklam. Bestäm också hur många biljetter som kommer att säljas samt hur mycket arrangören maximalt kan ha i fasta kostnader (hyra av lokal, ersättning till bandet, till vakter, etc.) för att arrangemanget inte skall gå med förlust.
\\
Tips: Vinstfunktionen $\pi(R) = 500(1000+4\sqrt{R})-R$, exklusive fasta kostnader. 
%</Question>

%<*Ans>
%</Ans>

\end{document}