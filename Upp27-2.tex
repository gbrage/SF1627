\documentclass[12pt]{article}
\usepackage[swedish]{babel}
\usepackage[utf8]{inputenc}                                  
\usepackage{enumitem, tasks}
\usepackage{amsmath,amssymb,amsthm, amsfonts}

\begin{document}
%<*Question>
Sök största och minsta värdet av funktion $f$ då
Ett företag kan sälja sin vara på två helt separerade marknader. Det är pris $P_1$ och $P_2$ som företaget kan ta ut på dessa marknader beror på hur mycket man säljer enligt
\begin{equation*}
P_1(Q_1) = 132-Q_1\text{ och } P_2(Q_2)=240-Q_2,
\end{equation*}
där $Q_1$ och $Q_2$ står för försälningsvolymer på respektive marknader. Den totala kostnaden för att producera varan visas av
\begin{equation*}
C(Q_1,Q_2)=10+(Q_1+Q_2)^2\, .
\end{equation*}
Antag att företaget vill maximera vinsten. Hur stor blir företagets produktion, hur mycket säljs på respektive marknad och till vilka priser?
\\
Tips: Vinstfunktionen $\pi (Q_1,Q_2) = P_1\cdot Q_1+P_2\cdot Q_2 - C$.
%</Question>

%<*Ans>
%</Ans>

\end{document}