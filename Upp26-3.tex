\documentclass[12pt]{article}
\usepackage[swedish]{babel}
\usepackage[utf8]{inputenc}                                  
\usepackage{enumitem, tasks}
\usepackage{amsmath,amssymb,amsthm, amsfonts}

\begin{document}
%<*Question>
För ett företag antas följande produktionsfunktion gälla:
\begin{equation*}
Q(K,L)=500\ln(KL^2)
\end{equation*}
där $K>1$, $L>1$. Företaget säljer sin vara till det givna priset $100:-$. Priset för kapital,$K$, är 100:- och för arbetskraft,$L$, 500:-. Hur stor är den maximala vinst som företaget kan uppnå och hur mycket kapital och arbetskraft skall företaget använda sig av för att uppnå denna vinst?
\\
Ledning: Vinstfunktionen ges av $\pi (K,L) = 100\cdot Q(K,L) - 100K-500L = 100\cdot 500\ln(KL^2)-100K-500L$. Maximera $\pi$.
%</Question>

%<*Ans>
%</Ans>

\end{document}