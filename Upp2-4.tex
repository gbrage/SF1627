


Ett företags totala kostnader $C$ för produktion av $x$ stycken av vara ges av funktionen $C(x) = x^3-120x^2+6000x $
\begin{enumerate}
\item Bestäm företagets marginalkostnad $C'(x)$.
\item Ge en ekonomisk tolkning av $C'(x)$. 
\item För vilket $x$-värde får man minsta marginalkostnad?
\item Beräkna $C'(40)$ och $C(41)-C(40)$, (går bra med räknedosa).
\end{enumerate}
Tips: Läs example 6.4.3-6.4.4 på sidan 180-181.