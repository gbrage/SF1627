\documentclass[12pt]{article}
\usepackage[swedish]{babel}
\usepackage[utf8]{inputenc}                                  
\usepackage{enumitem, tasks}
\usepackage{amsmath,amssymb,amsthm, amsfonts}

\begin{document}
%<*Question>
En rak cirkulär kon splaceras med spetsen vänd nedåt. Basytans radie är $R = 6$ m och höjden $H=8$ m. Konen fylls med vatten så att höjden på vattenytan stiger med hastigheten 1/3 m/min. Med vilken hastighet ökar volymen då vattendjupet är $h=2$ m?
\\
Tips: Vattens volym vid $t$ ges av $V(t) = \frac{\pi r^2(t)h(t)}{3}$, där $r(t)$ är vattenytans radie och $h(t)$ är vattenytans höjd. Man har $r(t) = \frac{ 3h(t)}{4}$, alltså $V(t) = \frac{3\pi h^3(t)}{16}$. Visöker $\frac{dV}{dt}$ då vi vet att $\frac{dh}{dt} = \frac{1}{3} $ och $h=2$ (vid den aktuella tiden).
%</Question>

Svar
%<*Ans>
%</Ans>

\end{document}