\documentclass[12pt]{article}
\usepackage[swedish]{babel}
\usepackage[utf8]{inputenc}                                  
\usepackage{enumitem, tasks}
\usepackage{amsmath,amssymb,amsthm, amsfonts}

\begin{document}
%<*Question>
Ett företag äger två produktionsanläggningar för att producera en och samma vara. Låt oss kalla produktionen i anläggning $X$ för $x$ och produktionen i anläggning $Y$ för $y$. Den totala kostnaden för produktion i respektive anläggning anges nedan:
\begin{equation*}
C_X(x) =20+x^2\quad\text{ och }\quad C_Y(y)=100y+0.5y^2\, .
\end{equation*}
\begin{tasks}
\task Antag att företaget skall producera en viss kvantitet $q$ till lägsta möjliga kostnad. Hur kommer företaget att fördela produktionen på sina två anläggningar? Använd Lagrangemultiplikator-metoden för att lösa problemet. Lösningen kommer då att visa optimala värden för $x$ och $y$ för godtyckliga värden på $q$. Förklara varför du ändå inte kan acceptera din lösning för alla värden på $q$, samt hur lösningen av företagets problem ser ut i dessa fall.
\task Antag att vi vet att den vinstmaximerande kvantiteten för företaget är 150 enheter. Hur skall produktionen mellan $X$ och $Y$ fördelas?
\end{tasks}
%</Question>

%<*Ans>
%</Ans>

\end{document}